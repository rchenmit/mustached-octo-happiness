\documentclass{article}

\begin{document}



\centerline{\sc \large 2 photon microscope scanpath generation and image processing}
\vspace{.5pc}
\today
\centerline{\sc Robert Chen}
%\centerline{\it (Read the .tex file along with this or it won't 
 %           make much sense)}
\vspace{2pc}

Currently most two-photon microscopes use the simple raster scan (scan back
and forth one line at a time) method to scan an image.  This method is well 
suited for imaging wide cell populations, it performs an entire scan in the 
scope's field of view in 1-2 minutes. The drawback of the raster scan method 
is that it takes a long time to scan an image, and does not allow for targeted
scanning of regions of interest (individual cells).   

We propose a method which allows for fast targeted scanning of individual regions of interest, as applied to the scanning
of individual neuron soma in a cortical slice.  The following method involves an initial raster scan of the entire field of
view, followed by an image recognition feature to detect regions of interest within the field of view. Next the regions of
interest are used to generate an intelligent scanning path connecting all regions of interst, creating an efficient scan 
path that allows the laser to target all regions of interest while traveling a minimal total distance through all of the 
regions.  This method allows us to maximize the amount of time imaging the regions of interest while minimizing the amount of 
time the scanner travels in between target regions.  Finally we incorporate a method to adjust the scanpath in response
to changes in the locations of regions of interest (perturbations of the cortical tissue, drifting of the focal view due to 
expansion of material in the scope).

\section{Method}

(see figure \ref{fig:methodOutline}) 
\begin{enumerate}
  \item Raster scan is performed for the entire field of view.\vspace{-2mm}%
  \item Cells detected.  Cells are indexed according to the centroid of the cells regions.\vspace{-2mm}%
  \item Calculate most efficient scan path which connects all cell centers.\vspace{-2mm}%
  \item Generate coordinates of the field of view which the scan path travels through.\vspace{-2mm}%
  \item Iterate through the following: (i) scan area with the calculated scanpath (ii) raster scan entire region every $t$ 
        change in time, detect cells, update scanpath based on new location of cells.%\vspace{-2mm}%
\end{enumerate}

\begin{figure}[htbp]
%%FIGURE HERE

\caption{Method Outline / Flow Chart}
\label{fig:methodOutline}
\end{figure}


\subsection{Calcium 2-photon microscope}
Two channels are used for imaging, the red channel and the green channel.  During preparation of the in vivo imaging, 
the red dye is injected into the cortical tissue. Red dye is absorbed by the itercellular space (ie. astrocytes, CSF, etc).
Green dye is injected afterwards and is absorbed by the soma of the neurons.

\subsection{Cell Detection Algorithm}
A cell detection algorithm is used to locate the centers of cells in an image.  The images from both the red channel and green
channel are converted into numerical matrices.  For our purpose we scan n frames (n = 34) from both channels.  The mean intensity
of the red frame (R) and the green (G) frame are taken. 
\begin{equation}
P = \frac{1}{n}\sum_{i=1}^{n}G_i-R_i.
\end{equation}

Next, P is filtered via the laplacian of the gaussian filter, in order to reduce the noise in the image. Afterwards the cells
are detected as the local maxima of the filtered image.


\subsection{Intelligent Scanpath}
A genetic algorithm is used to calculate the most efficient path through the set of points in two-dimensional space that 
are denoted by the cell centers that are detected in the previous step.  

\subsection{Conversion to coordinates for 2 photon}


Variation of algorithm from Lillis et. al. \cite{lillis}  With this algorithm, the scanner scans the entire cell, starting from one side and exiting at the opposite end.  While scanning 
inside a cell, the scanner moves slowly; while scanning between cells, the scanner accelerates, and slows down once it reaches the border of the next cell. 
(Figure \ref{fig:genpathOutput}).

A set of coordinates (locations) is generated, which represents the location on the field of view where the scanner will be for every interval $t$ time.  
Currently the algorithm generates a scanpath that can be scanned at 91.1079 Hz (default in Lillis et. al). \cite{lillis} The sample frequency $f_s$ is 125000 Hz,
while the sample period $dt = 1/f_s = 1/125000 = 8.0e-6$ seconds per pixel that is sampled.  The sample frequency and sample period refer to the time 
when the scanning is done inside cells.   


\begin{figure}[bp]
\begin{center}
\includegraphics[width=0.8\textwidth]{genpathxPosxVeloc30cells.eps}
\end{center}
\caption{Top: The calculated scanning path.  Area in red indicates time spent inside of a cell, area in blue indicates scanning between cells.  Middle: The x position of the scanning path
     Bottom: The x velocity of the scanning path.  Spikes indicate time spent between cells, when the scanner is accelerating from one cell to the next.}
\label{fig:genpathOutput}

\end{figure}


\subsection{Updating scanpath in response to change in cell location}
We have observed that there is a drift in the field of view for the microscope that occurs gradually over time.  This would occur for example when 
scanning the same area of brain tissue over a period of 30 minutes.  The rate of drift would be about $p$ pixels per second (see figure ref{fig:driftExample}).  

To account for the drift in the field of view, the scanpath will be updated every $time_{seconds}$ (seconds?) change in time. It is likely that the position of the cells will have shifted some distance,
relative to the field of view in the previous time point.  

We iterate through the following process for each new time point (each change in $time_{seconds}$.  

First, the cell detection algorithm is run, to locate the position of the cells.  Next the centroid of 
all $n$ cell locations is computed.  
Next let $P_t$ be the set of $n$ points that are present in the previous time point, and $P_{t-1}$ be the set of $n$ points that are in the current time point.  
\begin{equation}
C_t(x,y) = \frac{1}{n}\sum_{i=1}^{n}P_{(t,i)}(x_i, y_i).
\end{equation}

Let  
\begin{displaymath}
dC = C_t - C_{t-1}
\end{displaymath}
be the Euclidean distance change in centroid from previous time point to the current time point.  

Next, let $S = P_{t-1} + dC$ be the set of points in the previous time point, transformed by factor $dC$.  Now, we have shifted the points from the previous time point to the same field
of view as the current time point.  We will now locate the nearest points in set $P_t$ that correspond to set $S$, which is denoted by $M$.  If no point in set $P_t$ is found that is a distance
$d$ away from the corresponding point in $S$, $d$ being the average radius of a cell, then the corresponding point in $M$ is set equal to the transformed point in $S$.  

Now, $M$ will contain the set of points corresponding to the new location of cells in the current time point, in the same order as in $P_{t-1}$.  We then re-calculate the scanpath coordinates
based on these new locations of the cells.  (note: do not perform search for intelligent scanpath again - cells already in correct order).  


\section{Test case on 2-photon images}

Show a test case?
\\
\\
\\





\bibliographystyle{amsplain}
\bibliography{reflist}


\end{document}
